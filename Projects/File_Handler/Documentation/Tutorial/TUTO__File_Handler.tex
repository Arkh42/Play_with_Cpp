%%% PREAMBLE 
%%
% Document class
\documentclass[11pt,handout]{beamer}


% Font
\usepackage{fontspec}


% Typography
\usepackage{polyglossia}
\setdefaultlanguage{english}

\usepackage{csquotes}


% Beamer theme
\usetheme{Berkeley}


% Hyper links
\usepackage{hyperref}


% S I units
\usepackage[binary-units=true]{siunitx}

\sisetup{detect-all=true}
\sisetup{per-mode=symbol}


% References
\usepackage{cleveref}


% Doc info
\author{Alexandre \textsc{Quenon}}
\date{2017-10-21}% Created on
\title[C++ File Handler tutorial]{C++ project: File Handler tutorial}
%\institute{UMONS, F.P.Ms, SEMi / electroLAB}
%\logo{\includegraphics[height=5mm]{umons}}


% Automatic table of contents
\AtBeginSection[]
{
	\begin{frame}
		\frametitle{Overview}
		\tableofcontents[currentsection, hideothersubsections]
	\end{frame}
}



\begin{document}


\begin{frame}
	\titlepage
\end{frame}

\begin{frame}{Overview}
	\tableofcontents
\end{frame}


\section{Specifications}

	\subsection{Interests of specifications}
	
		\begin{frame}{Specifications}{The necessity of the specs}
			Specifications, or specs, allow to:
			\begin{itemize}
				\item clarify the context of a project,
				\item define the goals and the expectations,
				\item limit the scope of the desired features.
			\end{itemize}
		
			\begin{exampleblock}{Example: specs for a car engine}
				\begin{itemize}
					\item target: an engine
					\item context: for a car (not a plane nor a rocket)
					\item goals: minimize volume, minimize consumption\dots
					\item limit the scope: maximum power (e.g., so that speed $< \SI{150}{\km\per\hour}$)
				\end{itemize}
				
			\end{exampleblock}
		\end{frame}
		
		\begin{frame}{Specifications}{Types of specifications}
			There are two types of specifications for this project:
			\begin{enumerate}
				\item functional specifications,
				\item design specifications.
			\end{enumerate}
		
			\emph{Functional specs} define the aims as well as the different functions, or blocks, of the whole system.
			
			\emph{Design specs} describe the implementation of the functions, i.e., the structure, architecture, design techniques as well as specific features.
		\end{frame}
	
	
	\subsection{Specifications of \enquote{File Handler}}
	
		\begin{frame}{Functional specifications of \enquote{File Handler}}{What is included}
			Functional specifications must define:
			\begin{itemize}
				\item the aims of the project
				\item the interface of the end-user
			\end{itemize}
		\end{frame}
	
		\begin{frame}{Functional specifications of \enquote{File Handler}}{Aims of the project}
			Handling files refers to:
			\begin{itemize}
				\item automatically opening and closing files (RAII),
				\item opening either for read or write operations according to the request of the user,
				\item allow access to only one thread at a time.
			\end{itemize}
			
			\begin{alertblock}{Thread safe operation}
				File stream use $=$ critical resource access in multi-thread application $\Longrightarrow$ control access strategy is required!
			\end{alertblock}
		\end{frame}
	
		\begin{frame}{Functional specifications of \enquote{File Handler}}{End-User Interface}
			Two types of interface:
			\begin{enumerate}
				\item read operations,
				\item write operations.
			\end{enumerate}
		
			Conversely, some specific behaviours must be hidden to the end-user, i.e., must be excluded from the interface!
		\end{frame}
	
		\begin{frame}{Design specifications of \enquote{File Handler}}
			xxx
		\end{frame}
	
	
	
\end{document}
