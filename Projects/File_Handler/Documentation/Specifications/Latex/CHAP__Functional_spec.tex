\chapter{Functional specifications}

Functional specifications define the aims as well as the different functions, or blocks, of the whole system.


\section{Context}

	One of the most important processes in code design flow is testing.
	It is so important that some programmers embed testing in their design, using a technique named \enquote{Test-Driven Development}.
	
	Testing requires providing information to the designer.
	Information can be given by outputting the results directly in the terminal.
	However, the test must be run and there is no possibility to track the results from a version to another.
	Hence, writing the test results in a file is almost mandatory.
	Besides the context of C++ program testing, most of coding projects require files writing and reading.
	In order to do so, a \enquote{file handler} tool would be useful.
	


\section{Aims}

	The current project targets two generic aims:
	\begin{enumerate}
		\item achieve C++ programming features for files handling,
		\item learn specific C++ programming techniques.
	\end{enumerate}


	\subsection{C++ features to achieve}

	 	As the \enquote{File\_Handler} name suggests, the aim of the current project consists in \emph{handling files}.
	 	Handling refers to:
	 	\begin{itemize}
	 		\item automatically opening and closing files (RAII class),
	 		\item opening either for read or write operations according to the request of the user,
	 		\item allow access to only one thread at a time.
	 	\end{itemize}
 	
	 	However, \textbf{the handler is not an interpreter} of the file content: the \enquote{File\_Handler} reads and writes but does not understand!
	 	As a consequence, it cannot be used to extract information.
	 	
	 	
	 \subsection{C++ programming techniques to learn}
	 
	 	Handling files corresponds to \emph{resource management}.
	 	During the execution of a program, every necessary resource must be  acquired before being used, and released properly to prevent it from being corrupted.
	 	One possible implementation of resource management is the RAII technique.
	 	\textbf{RAII}, which stands for \enquote{Resource Acquisition Is Initialization}, is an object-oriented programming technique which consists in acquiring the resource at object construction and releasing it at destruction, making the resource tied to the object lifetime
	 	
	 	In addition, in the case of a multi-thread application, a resource can be turned into a \emph{critical resource}, which implies that either no simultaneous access is allowed or only a maximum number of threads can access the resource simultaneously.
	 	File writing does not allow simultaneous access.
	 	Otherwise, a critical race will occur and the data will be mixed in an unpredictable way.
	 	As a consequence, a control of the resource access must be implemented.
 	
 
 
\section{End-User Interface}

	The \enquote{File\_Handler} shall provide two types of interface:
	\begin{enumerate}
		\item read operations,
		\item write operations.
	\end{enumerate}
	Conversely, specific features must absolutely be hidden to the end-user.
	Hence, they must not be part of the interface.
	
	Expected interface for the read operations:
	\begin{itemize}
		\item 
	\end{itemize}

	Expected interface for the write operations:
	\begin{itemize}
		\item 
	\end{itemize}

	Hidden features that are not part of the end-user interface:
	\begin{itemize}
		\item resource management (acquisition, request of use and release);
		\item strategy for thread safe multiple access.
	\end{itemize}
	