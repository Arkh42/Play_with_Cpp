\chapter{Functional specifications}

Functional specifications define the aims as well as the different functions, or blocks, of the whole system.


\section{Aims}

 	As the \enquote{File\_Handler} name suggests, the aim of the current project consists in \emph{handling files}.
 	Handling refers to:
 	\begin{itemize}
 		\item automatically opening and closing files (RAII\footnote{RAII, which stands for \enquote{Resource Acquisition Is Initialization}, is an object-oriented programming technique which consists in acquiring the resource at object construction and releasing it at destruction, making the resource tied to the object lifetime} class),
 		\item opening either for read or write operations according to the request of the user,
 		\item allow access to only one thread at a time.
 	\end{itemize}
 	\textcolor{red}{TBC!} However, there is \textbf{no interpretation} of the file content: the \enquote{File\_Handler} reads and writes but does not understand!
 
 	In the case of a multi-thread application, the use of any file stream is equivalent to an access to a \emph{critical resource}.
 	Indeed, if different threads use simultaneously the same stream, a critical race will occur and the data will be mixed in an unpredictable way.
 	In order to prevent such an event from occurring, a control of the resource access must be implemented.
 	
 
 
\section{End-User Interface}

	The \enquote{File\_Handler} shall provide two types of interface:
	\begin{enumerate}
		\item read operations,
		\item write operations.
	\end{enumerate}
	Conversely, specific features must absolutely be hidden to the end-user.
	Hence, they must not be part of the interface.
	
	Expected interface for the read operations:
	\begin{itemize}
		\item 
	\end{itemize}

	Expected interface for the write operations:
	\begin{itemize}
		\item 
	\end{itemize}

	Hidden features that are not part of the end-user interface:
	\begin{itemize}
		\item resource management (acquisition, request of use and release);
		\item strategy for thread safe multiple access.
	\end{itemize}
	