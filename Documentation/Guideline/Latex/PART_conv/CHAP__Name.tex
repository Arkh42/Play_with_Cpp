\chapter{Naming stuff}

The purpose of the current chapter is to provide generic as well as specific rules in order to have an easy-to-read code.
This if probably one of the most important part, especially when working in teams, because it helps you to track bugs and to understand each other's code.



\section{Generic guideline}

Let us start with some generic rules that can be applied in the whole code, whatever we are working on a class or a function.


\begin{guiderule}[Use explicit names that express ideas, concepts.]
	Compare phone\_book with number\_vector \cite{Stroustrup_2013__Cpp_Prog_Lang,Martin_2009__Coder_propr}.
	
	C++ is object-oriented programming.
	Ideas and concepts have a higher level of abstraction.
\end{guiderule}


\begin{guiderule}[Prefer underscore to camel case for readability.]
	\label{RUL::Name:underscore}
	Compare number\_of\_element with numberOfElement \cite{Stroustrup_2013__Cpp_Prog_Lang}.
\end{guiderule}


\begin{guiderule}[In general, avoid capital letters.]
	\label{RUL::Name:avoid_capitals}
	It is simply easier and faster to write non-capital letters.
\end{guiderule}


\Cref{RUL::Name:underscore,RUL::Name:avoid_capitals} are followed by the Standard Library.



\section{Specific guidelines}

Let us now review some rules that are intended for specific part of the codes, such as variables or functions.
Of course, we may apply them for other things but, in general, we will mainly face the situation for the specified subject.


	\subsection{Variables}
	
		\begin{guiderule}[Do not embed the type in the name.]
			It lowers the abstraction level.
			Moreover, embedding the type in the variable name is a nightmare of maintenance and consistency.
			Suppose you decide to change your \verb|int| to a \verb|double|, how many names will you have to change to maintain the consistency? \cite{Martin_2009__Coder_propr,Stroustrup_2013__Cpp_Prog_Lang}
			
			Examples: do not use something like \verb|char* pc_buffer| or \verb|int i_number|.
		\end{guiderule}
	
		The previous rules could also be applied to functions (about their return type, for instance).
		However, this is typically done for variables by programmers who are used to weak-typed languages.


	\subsection{User-defined types (\texttt{class}es and \texttt{enum}s)}
	
		\begin{optionalrule}[Capitalize names for user-defined types.]
			\label{RUL::Name:user_defined}
			Examples: Widget, My\_Super\_Class.
		\end{optionalrule}
	
		For your information, \cref{RUL::Name:user_defined} is not followed by the Standard Library (e.g., \verb|std::vector| or \verb|std::string|).
		However, to my opinion, it is easier for the programmer to identify variables, return types\dots that are of built-in type or part of the STL from their own classes or enumerations.
		Of course, it is an exception to \cref{RUL::Name:avoid_capitals}.
		Eventually, following this rule or not is your choice.
	
	
	\subsection{Macros}
	
		\begin{guiderule}[Use all-capital names for macros.]
			\label{RUL::Name:macro}
			Examples: CRC\_CODE, BUFSIZE.
		\end{guiderule}
	
		\textcolor{red}{TBC}
		Macros are more C-like programming.
		As a consequence, their use should be seen as something unusual.
		Therefore, to draw readers' attention on any used macro, \cref{RUL::Name:macro} is not named optional.
		This is an intentional exception to \cref{RUL::Name:avoid_capitals} because using macros should be annoying.

